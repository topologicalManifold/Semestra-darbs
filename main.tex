\documentclass{article}
\usepackage[utf8]{inputenc}
\usepackage{hyperref}

\hypersetup{
    colorlinks=true,
    linkcolor=blue,
    filecolor=blue,      
    urlcolor=blue,
}

\title{Lēts Photoshop}
\author{Marta Kalniņa, Antons Nikolajevs }
\date{Decembris 2020}

\begin{document}

\maketitle

\section{Uzdevuma nosacījumi}
Izveidot programmu ar grafisko saskarni, ar kuras palīdzību var ielasīt attēlus un ar peles kursoru iezīmēt kādu attēla apgabalu. Programma veic iezīmētās attēla daļas krāsu statistisko analīzi: parāda pikseļu gaišuma (luminosity) histogrammu un vidējo krāsu.
\section{Grafiskais interfeiss}
Grafiskā interfeisa izveidei tika izmantota \href{https://kivy.org/#home}{Kivy} bibliotēka. Diemžēl, logdaļas (widgets), ko piedāvā Kivy izskatās slikti\footnote{tie izskatās tā, itkā būtu biedotas, kad Windows 95 bija poppulārs}.

\section{Attēla apgabala iezīmēšana}
Uzdevuma nosacījumos tiek prasīts, lai lietotājs varētu iezīmēt attēla apgabalu, lai veiktu tā krāsu statistisko analīzi.
Tā kā tālākā attēla apstrāde notiek, izmantojot \textit{PIL} moduli, kur attēla koordinātas tiek skaitītas no augšējā kreisajā stūra (x ass ir vērsta pa labi un y ass ir vērsta uz leju), ir nepieciešams pārveidot vienu koordinātu sistēmu uz otru.

\section{Vidējās krāsas noteikšana}

Katra pikseļa krāsu nosaka trīs skaitļi no 0 līdz 255. Katrs skaitlis atbilst attiecīgas krāsas - sarkanās, zaļas un zilas (RGB) - īpatsvaru. Attēls (vai arī attēla iezīmētais apgabals), kura izmēri ir $n\times m$ pikseļi tiek glabāts kā $n\times m \times 3$ vai arī $n \times m \times 4$ (jā attēls ir RGBA\footnote{RGBA darbojas ļoti līdzīgi, kā RGB, bet tam ir vēl viens papildus skaitlis no 0 līdz 255 katram pikselim, kas nosaka pikseļa caurspīdīgumu} formātā) matrica, kuru turpmāk apzīmēsim $A$ un atsevišķa pikseļa konkrētas krāsas intensitāti apzīmēsim ar $A_{ijk}$. Piemēram, pikseļa $(6, 12)$ sarkanās krāsas intensitāte ir $A_{6,12,1}$. Līdz ar to, lai noteiktu vidējo krāsu, ir jānosaka katras pamatkrāsas (sarkanās, zaļas un zilas) vidējo intensitāti. To var izdarīt šādi:
$$
    C_{k} = \frac{\sum_i^n \sum_j^m A_{i,j,k}}{n\cdot m}
$$
Sākumā var atrast vidējo vērtību katrā matricas (attēla) rindiņā $ R_{i,k} = \frac{1}{m}\sum_j^m A_{i,j,k}$, un tad no tā atrast vidējo vērtību starp rindiņu vedējām vērtībām:
$$
    C_{k} = \frac{1}{n}\sum_i^n R_{i,k}
$$
Tā, kā rezultātā vispārīgajā gadījumā vērtība $C{k}$ ir daļskaitlis, tas jānoapaļo, jo tam jābūt veselajam skaitlim no 0 līdz 255.
\section{Pikseļu gaišuma histogramma}


\end{document}
